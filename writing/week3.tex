\chapter{Discrete Fourier Transform, Windowing and Spectral Leakage}
% https://www.geeksforgeeks.org/software-engineering/discrete-fourier-transform-and-its-inverse-using-matlab/
%  https://www.tutorialspoint.com/discrete-fourier-transform-and-its-inverse-using-matlab
% https://warwick.ac.uk/fac/sci/moac/people/students/2005/jing_kang/dsp/seminar6_dft_matlab.pdf


\section{Learning objectives}

After completing this chapter you will be able to:
\begin{itemize}
  \item Compute the discrete Fourier transform (DFT) and interpret its magnitude and phase spectra.
  \item Explain the effects of windowing on spectral leakage and choose appropriate window functions for different tasks.
  \item Apply the Fast Fourier Transform (FFT) as a black box and tune parameters such as segment length and zero‑padding.
  \item Quantify leakage using metrics such as the leakage ratio and design experiments to demonstrate leakage.
\end{itemize}

\section{Industrial context}
The Discrete Fourier Transform (DFT) and its Inverse (IDFT) are core techniques in digital signal processing. They convert signals between the time or spatial domain and the frequency domain, revealing frequency components in data.

Frequency analysis is fundamental in machinery diagnostics, audio processing and communications.  Engineers use the FFT to identify harmonics, modulations and fault signatures.  However, the finite observation window introduces artefacts: leakage spreads energy into adjacent bins and can mask weak components.  Understanding and mitigating leakage is essential when monitoring rotating machinery or analysing speech signals.

\section{Core concepts}

\subsection{Discrete Fourier transform}

The standard equations which define how the Discrete Fourier Transform and the Inverse convert a signal from the time domain to the frequency domain and vice versa are as follows: 

For a sequence $x[n]$ of length $N$, the DFT and its inverse are defined by
\[
X[k] = \sum_{n=0}^{N-1} x[n] e^{-j 2\pi nk/N},\qquad
x[n] = \frac{1}{N} \sum_{k=0}^{N-1} X[k] e^{j2\pi nk/N}
\].
The FFT is an efficient algorithm that computes the DFT in $O(N\log N)$ time.

\subsection{Windowing and leakage}

When we observe a signal through a finite window, we multiply it by a window function $w[n]$.  Multiplication in time corresponds to convolution in frequency, smearing the spectrum.  Rectangular windows have narrow main lobes but high side lobes, leading to leakage.  Tapered windows (Hann, Hamming, Blackman) suppress side lobes at the cost of a wider main lobe.  The leakage ratio in the demonstration is defined as the ratio of the largest spectral component outside the main lobe to the peak inside the main lobe.

\section{Operational formulas}

\paragraph{Zero‑padding.}  Appending zeros to a time‑domain signal before computing the FFT increases the apparent frequency resolution (interpolates the spectrum) but does not improve the ability to separate closely spaced frequencies.

\paragraph{Window choice.}  The 3 dB bandwidth of a Hann window is approximately $2/N$ (in normalised frequency).  To resolve two tones separated by $\Delta f$, choose a window length $N$ such that $f_s/N < \Delta f$.

\section{Parameter tuning playbook}

\begin{table}[h]
  \centering
  \begin{tabular}{@{}llll@{}}
    \toprule
    \textbf{Knob} & \textbf{Default} & \textbf{Symptom} & \textbf{Adjustment} \\
    \midrule
    Window type & Hann & Leakage obscures weak tones & Use Blackman for lower side lobes \\
    Segment length $N$ & 1024 & Frequency bins too coarse & Increase $N$ (longer time record) \\
    Zero‑padding factor & 1 & Peak locations between bins & Increase zero‑padding to refine peak estimate \\
    Overlap & 50\% & Excessive variance between spectra & Increase overlap or average multiple segments \\
    \bottomrule
  \end{tabular}
  \caption{Parameter tuning guidelines for FFT analysis and windowing.}
\end{table}

\section{Pitfalls, failure modes and diagnostics}

\begin{itemize}
  \item \textbf{Spectral leakage.}  Leakage spreads energy into side lobes.  Mitigate by using tapered windows and ensuring the signal contains an integer number of cycles within the window.
  \item \textbf{Scalloping loss.}  When a tone falls between FFT bins, the measured magnitude decreases.  Windowing reduces scalloping loss at the expense of resolution.
  \item \textbf{Aliasing.}  If the sampled signal contains components above the Nyquist frequency, the FFT shows aliased peaks.  Use anti‑alias filtering (see Chapter 2).
\end{itemize}

\section{Code walkthrough}

The Week\,3 demonstration generates a two‑tone signal and computes the FFT using different windows.  It calculates the leakage ratio as the ratio of power outside the main lobe to the power inside.  The script highlights how a Hann window reduces leakage compared with a rectangular window.

\section{Exercises}

\begin{enumerate}
  \item Derive the relationship between the window length and the Rayleigh frequency $1/T$ (where $T$ is the window duration).  For a given sampling rate and desired frequency resolution, choose an appropriate window length.
  \item Compare the leakage ratios for Hann, Hamming and Blackman windows on a two‑tone signal.  Explain the trade‑off between main‑lobe width and side‑lobe level.
  \item Modify the demonstration to include zero‑padding and observe how it improves the apparent frequency resolution without changing the ability to separate two tones.
\end{enumerate}

\section{References}

\begin{itemize}
  \item DFT and inverse DFT formulas 【745517071471635†L44-L49】.
\end{itemize}



%%%%%%%%%% PROMP


% You are co-authoring a textbook chapter for an undergraduate course: “Digital Signal Processing for Industrial Physics.”

% Writing rules (non-negotiable):
% - Black-box philosophy: focus on (1) required inputs/assumptions, (2) parameter choices and tuning, (3) how to interpret outputs, (4) failure modes and diagnostics, (5) industrial relevance.
% - Avoid proofs/derivations. Keep math minimal and operational (short formulas are OK if used as “rules to apply”).
% - Always connect FFT results to real instrumentation realities: sensor noise, finite records, nonstationarity, speed drift, quantization, saturation, EMI, and timing jitter.

% Task:
% Expand and restructure the LaTeX chapter “Discrete Fourier Transform, Windowing and Spectral Leakage” to be more informative.
% Keep the existing sections (Learning objectives, Industrial context, Core concepts, Operational formulas, Parameter tuning playbook, Pitfalls, Code walkthrough, Exercises), but add practical subsections that practitioners need:

% Required additions:
% 1) A “frequency-axis sanity” section: how to compute the frequency bin spacing, Nyquist, one-sided vs two-sided spectra, and correct amplitude scaling (peak vs RMS; window coherent gain).
% 2) Windowing as a tool: when to use rectangular vs Hann/Hamming/Blackman/Kaiser; interpret main-lobe width vs side-lobe level.
% 3) Leakage mechanics explained operationally: integer cycles, non-integer cycles, and what leakage looks like; define a leakage metric clearly.
% 4) Resolution vs record length: Rayleigh resolution (as a rule), and why zero-padding doesn’t create new resolution.
% 5) FFT as a measurement instrument: how to choose N, overlap, and averaging (Welch) for noisy industrial signals.
% 6) Peak finding and frequency estimation: bin interpolation (quadratic) and how zero-padding helps peak localization.
% 7) A short case study (with numbers) for rotating machinery or power electronics: what settings you pick, what goes wrong, and how you fix it.
% 8) A “diagnostics checklist” that a technician can follow when the spectrum looks suspicious.

% Output format:
% - First provide an improved chapter outline (sections/subsections).
% - Then provide expanded LaTeX content for each section (not just bullets). Include at least:
%   - one “Quick Recipe” box (step-by-step),
%   - one “Decision tree” or checklist,
%   - a “Parameter tuning playbook” table (you can reuse and expand the existing table),
%   - clear definitions of metrics (leakage ratio, scalloping loss, noise floor estimate).
% - Include a Python walkthrough plan (NumPy/SciPy) that compares windows and shows quantitative metrics, plus a parameter sweep over N and overlap.
% - Finish with industrially grounded exercises that test operational understanding (no proofs).
% ```

% ---

% ## 2) Suggested expanded chapter structure (more informative, still black-box)

% This structure keeps your original sections but adds the missing “how to use FFT in the field” pieces.

% ### 0. Chapter map and “FFT as an instrument” (new, short)

% * What FFT can tell you reliably vs what it can’t.
% * The minimum plots/checks you should always do.

% ### 1. Learning objectives (keep, add success criteria)

% Add “you can do it if…” lines:

% * “Given (f_s) and (N), you can label the frequency axis correctly.”
% * “Given a tone between bins, you can predict scalloping loss and choose a window.”

% ### 2. Industrial context (expand into mini vignettes)

% * Rotating machinery: harmonics, sidebands, speed drift.
% * Power electronics: switching ripple, EMI spikes, harmonics.
% * Audio/acoustics: weak tones near strong components, leakage masking.

% ### 3. Core concepts (expand into practical building blocks)

% #### 3.1 DFT/FFT outputs and how to read them (new emphasis)

% * Two-sided vs one-sided spectra (real signals).
% * Frequency bins: spacing (\Delta f = f_s/N), Nyquist, bin index → Hz.
% * Magnitude/phase interpretation (what phase is meaningful for).

% #### 3.2 Amplitude scaling and window coherent gain (new)

% * Why “FFT magnitude” is not automatically amplitude.
% * Coherent gain correction for windows (operational: “divide by CG”).
% * Peak vs RMS conventions; dB scaling.

% #### 3.3 Windowing and leakage (keep, expand)

% * Leakage as “finite record mismatch.”
% * Integer cycles vs non-integer cycles.
% * Main-lobe width vs side-lobes as a trade.
% * Define leakage ratio precisely and show how to measure it.

% #### 3.4 Resolution vs variance: record length, overlap, averaging (new)

% * Rule: longer (N) → finer (\Delta f), but slower time tracking.
% * Welch averaging reduces variance, not leakage.
% * Overlap as a knob for smoother PSD estimates.

% #### 3.5 Zero-padding and peak localization (keep, expand)

% * Zero-padding interpolates the spectrum (helps peak picking).
% * Bin interpolation (quadratic) for frequency estimates.

% ### 4. Operational formulas (keep but turn into “rules you apply”)

% * Frequency axis rule: (f[k]=k f_s/N).
% * One-sided scaling rule for real signals.
% * Rayleigh “resolution” rule: (\approx 1/T) (operational).
% * Leakage ratio definition and scalloping loss reminder.

% ### 5. Parameter tuning playbook (expand your table + add validation steps)

% Add knobs like:

% * detrending / mean removal,
% * window normalization,
% * averaging method (Welch),
% * dynamic range tricks (log scale, PSD vs magnitude).

% ### 6. Pitfalls, failure modes and diagnostics (expand into a workflow)

% * Leakage, scalloping, aliasing (keep).
% * Added: wrong frequency axis, wrong scaling, speed drift smearing, nonstationarity, saturation harmonics, DC offsets.
% * A step-by-step “suspicious spectrum” checklist.

% ### 7. Mini case study (new)

% Example: 1× and 2× shaft frequency + a weak bearing tone near a strong harmonic.

% * Show wrong setup (rectangular, short record) → leakage hides weak tone.
% * Fix (Hann/Blackman + longer record + Welch) → weak tone emerges.
% * Include numbers: (f_s), (N), overlap, window, leakage metric.

% ### 8. Code walkthrough (keep, make it reproducible narrative)

% * Build signal + noise + speed drift variant.
% * Compare windows quantitatively (leakage ratio, scalloping loss).
% * Sweep (N), overlap → show variance vs resolution trade.

% ### 9. Exercises (revise to be operational, not derivational)

% Replace “derive relationship” with “verify/measure relationship” using plots + metrics.


