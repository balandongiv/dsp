\chapter{ADC and DAC}
Most of the signals directly encountered in science and engineering are continuous: light intensity that changes with distance; voltage that varies over time; a chemical reaction rate that depends on temperature, etc. Analog-to-Digital Conversion (ADC) and Digital-to-Analog Conversion (DAC) are the processes that allow digital computers to interact with these everyday signals. Digital information is different from its continuous counterpart in two important respects: it is sampled, and it is quantized. Both of these restrict how much information a digital signal can contain. This chapter is about information management: understanding what information you need to retain, and what information you can afford to lose. In turn, this dictates the selection of the sampling frequency, number of bits, and type of analog filtering needed for converting between the analog and digital realms. 

\section{Quantization}

% source: http://www.dspguide.com/ch3/1.htm
First, a bit of trivia. As you know, it is a digital computer, not a digit computer. The information processed is called digital data, not digit data. Why then, is analog-to-digital conversion generally called: digitize and digitization, rather than digitalize and digitalization? The answer is nothing you would expect. When electronics got around to inventing digital techniques, the preferred names had already been snatched up by the medical community nearly a century before. Digitalize and digitalization mean to administer the heart stimulant digitalis.

Figure 3-1 shows the electronic waveforms of a typical analog-to-digital conversion. Figure (a) is the analog signal to be digitized. As shown by the labels on the graph, this signal is a voltage that varies over time. To make the numbers easier, we will assume that the voltage can vary from 0 to 4.095 volts, corresponding to the digital numbers between 0 and 4095 that will be produced by a 12 bit digitizer. Notice that the block diagram is broken into two sections, the sample-and-hold (S/H), and the analog-to-digital converter (ADC). As you probably learned in electronics classes, the sample-and-hold is required to keep the voltage entering the ADC constant while the conversion is taking place. However, this is not the reason it is shown here; breaking the digitization into these two stages is an important theoretical model for understanding digitization. The fact that it happens to look like common electronics is just a fortunate bonus.

As shown by the difference between (a) and (b), the output of the sample-and-hold is allowed to change only at periodic intervals, at which time it is made identical to the instantaneous value of the input signal. Changes in the input signal that occur between these sampling times are completely ignored. That is, sampling converts the independent variable (time in this example) from continuous to discrete.

As shown by the difference between (b) and (c), the ADC produces an integer value between 0 and 4095 for each of the flat regions in (b). This introduces an error, since each plateau can be any voltage between 0 and 4.095 volts. For example, both 2.56000 volts and 2.56001 volts will be converted into digital number 2560. In other words, quantization converts the dependent variable (voltage in this example) from continuous to discrete.

Notice that we carefully avoid comparing (a) and (c), as this would lump the sampling and quantization together. It is important that we analyze them separately because they degrade the signal in different ways, as well as being controlled by different parameters in the electronics. There are also cases where one is used without the other. For instance, sampling without quantization is used in switched capacitor filters.

First we will look at the effects of quantization. Any one sample in the digitized signal can have a maximum error of ±? LSB (Least Significant Bit, jargon for the distance between adjacent quantization levels). Figure (d) shows the quantization error for this particular example, found by subtracting (b) from (c), with the appropriate conversions. In other words, the digital output (c), is equivalent to the continuous input (b), plus a quantization error (d). An important feature of this analysis is that the quantization error appears very much like random noise.

This sets the stage for an important model of quantization error. In most cases, quantization results in nothing more than the addition of a specific amount of random noise to the signal. The additive noise is uniformly distributed between ±? LSB, has a mean of zero, and a standard deviation of 1/√12 LSB (~0.29 LSB). For example, passing an analog signal through an 8 bit digitizer adds an rms noise of: 0.29/256, or about 1/900 of the full scale value. A 12 bit conversion adds a noise of: 0.29/4096 ≈ 1/14,000, while a 16 bit conversion adds: 0.29/65536 ≈ 1/227,000. Since quantization error is a random noise, the number of bits determines the precision of the data. For example, you might make the statement: "We increased the precision of the measurement from 8 to 12 bits."

This model is extremely powerful, because the random noise generated by quantization will simply add to whatever noise is already present in the 

\section{The Sampling Theorem}
\section{Digital-to-Analog Conversion}
\section{Analog Filters for Data Conversion}