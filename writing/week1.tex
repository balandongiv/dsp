\chapter{Sampling, Quantisation and Information Budgets}

\section{Learning objectives}

After studying this chapter you will be able to:
\begin{itemize}
  \item Explain what is signal
\end{itemize}

\section{Sounds and signals}


A signal represents a quantity that varies in time. That definition is pretty abstract, so let’s start with a concrete example: sound. Sound is variation in air pressure. A sound signal represents variations in air pressure over time.

A microphone is a device that measures these variations and generates an electrical signal that represents sound. A speaker is a device that takes an electrical signal and produces sound. Microphones and speakers are called transducers because they transduce, or convert, signals from one form to another.

This book is about signal processing, which includes processes for synthesizing, transforming, and analyzing signals. I will focus on sound signals, but the same methods apply to electronic signals, mechanical vibration, and signals in many other domains.

They also apply to signals that vary in space rather than time, like elevation along a hiking trail. And they apply to signals in more than one dimension, like an image, which you can think of as a signal that varies in two-dimensional space. Or a movie, which is a signal that varies in two-dimensional space and time.

But we start with simple one-dimensional sound.
\section{What is Digital Signal Processing}


\subsection{The Root of DSP}

Digital Signal Processing is distinguished from Other areas in computer science
by the unique type Of data it uses: signals. In most cases, these signals
originate as sensory data from the real world: seismic vibrations, visual images,
sound waves, etc. DSP is the mathematics, the algorithms, and the techniques
used to manipulate these signals after they have been converted into a digital
form. This includes a wide variety of goals, such as: enhancement of visual
images, recognition and generation Of speech, compression Of data for storage
and transmission, etc. Suppose we attach an analog-to-digital converter to a
computer and use it to acquire a chunk of real world data. DSP answers the
question: What next?
The roots of DSP are in the 1960s and 1970s when digital computers first
became available. Computers were expensive during this era, and DSP was
limited to only a few critical applications. Pioneering efforts were made in four
key areas: radar & sonar, where national security was at risk; oil exploration,
where large amounts Of money could be made; space exploration, where the data are irreplaceable; and medical imaging, where lives could be saved.
The personal computer revolution Of the 1980s and 1990s caused DSP to
explode with new applications. Rather than being motivated by military and
government needs, DSP was suddenly driven by the commercial marketplace.
Anyone who thought they could make money in the rapidly expanding field was
suddenly a DSP vender. DSP reached the public in such products as: mobile
telephones, compact disc players, and electronic voice mail. Figure 1-1
illustrates a few of these varied applications.
This technological revolution occurred from the top-down. In the early
1980s, DSP was taught as a graduate level course in electrical engineering.
A decade later, DSP had become a standard part Of the undergraduate
curriculum. Today, DSP is a basic skill needed by scientists and engineers in many fields. As an analogy, DSP can be compared to a previous
technological revolution: electronics. While still the realm of electrical
engineering, nearly every scientist and engineer has some background in basic
circuit design. Without it, they would be lost in the technological world. DSP
has the same future.
This recent history is more than a curiosity; it has a tremendous impact on your
ability to learn and use DSP- Suppose you encounter a DSP problem, and turn
to textbooks or other publications to find a solution. What you will typically
find is page after page of equations, obscure mathematical symbols, and
unfamiliar terminology. It's a nightmare! Much Of the DSP literature is
baffling even to those experienced in the field. It's not that there is anything
wrong with this material, it is just intended for a very specialized audience.
State-of-the-art researchers need this kind of detailed mathematics to
understand the theoretical implications of the work.
A basic premise of this book is that most practical DSP techniques can be
learned and used without the traditional barriers of detailed mathematics and
theory. The Scientist and Engineer 's Guide to Digital Signal Processing is
written for those who want to use DSP as a tool, not a new career.
The remainder of this chapter illustrates areas where DSP has produced
revolutionary changes. As you go through each application, notice that DSP
is very interdisciplinary, relying on the technical work in many adjacent
fields. As Fig. 1-2 suggests, the borders between DSP and other technical
disciplines are not sharp and well defined, but rather fuzzy and overlapping.
If you want to specialize in DSP, these are the allied areas you will also
need to study.

\section{Example of Application}
\subsection{Telecommunications}
Telecommunications is about transferring information from one location to
another. This includes many forms of information: telephone conversations,
television signals, computer files, and other types of data. To transfer the
information, you need a channel between the two locations. This may be
a wire pair, radio signal, optical fiber, etc. Telecommunications companies
receive payment for transferring their customer's information, while they
must pay to establish and maintain the channel. The financial bottom line
is simple: the more information they can pass through a single channel, the
more money they make. DSP has revolutionized the telecommunications
industry in many areas: signaling tone generation and detection, frequency
band shifting, filtering to remove power line hum, etc. Three specific
examples from the telephone network will be discussed here: multiplexing,
compression, and echo control.

\subsubsection{Multiplexing}

There are approximately one billion telephones in the world. At the press of
a few buttons, switching networks allow any one of these to be connected to
any other in only a few seconds. The immensity of this task is mind boggling!
Until the 1960s, a connection between two telephones required passing the
analog voice signals through mechanical switches and amplifiers. One
connection required one pair of wires. In comparison, DSP converts audio
signals into a stream of serial digital data. Since bits can be easily
intertwined and later separated, many telephone conversations can be
transmitted on a single channel. For example, a telephone standard known
as the T-carrier system can simultaneously transmit 24 voice signals. Each
voice signal is sampled 8000 times per second using an 8 bit companded
(logarithmic compressed) analog-to-digital conversion. This results in each
voice signal being represented as 64,000 bits/sec, and all 24 channels being
contained in 1.544 megabits/sec. This signal can be transmitted about 6000
feet using ordinary telephone lines of 22 gauge copper wire, a typical
interconnection distance. The financial advantage of digital transmission
is enormous. Wire and analog switches are expensive; digital logic gates
are cheap.

\subsubsection{Compression}

When a voice signal is digitized at 8000 samples/sec, most of the digital
information is redundant. That is, the information carried by any one
sample is largely duplicated by the neighboring samples. Dozens of DSP
algorithms have been developed to convert digitized voice signals into data
streams that require fewer bits/sec. These are called data compression
algorithms. Matching uncompression algorithms are used to restore the
signal to its original form. These algorithms vary in the amount of
compression achieved and the resulting sound quality. In general, reducing the
data rate from 64 kilobits/sec to 32 kilobits/sec results in no loss of sound
quality. When compressed to a data rate of 8 kilobits/sec, the sound is
noticeably affected, but still usable for long distance telephone networks.
The highest achievable compression is about 2 kilobits/sec, resulting in sound that is highly distorted, but usable for some applications such as military
and undersea communications.

\subsubsection{Echo Control}

Echoes are a serious problem in long distance telephone connections.
When you speak into a telephone, a signal representing your voice travels
to the connecting receiver, where a portion of it returns as an echo. If the
connection is within a few hundred miles, the elapsed time for receiving the
echo is only a few milliseconds. The human ear is accustomed to hearing
echoes with these small time delays, and the connection sounds quite
normal. As the distance becomes larger, the echo becomes increasingly
noticeable and irritating. The delay can be several hundred milliseconds
for intercontinental communications, and is particularity objectionable.
Digital Signal Processing attacks this type of problem by measuring the
returned signal and generating an appropriate antisignal to cancel the
offending echo. This same technique allows speakerphone users to hear
and speak at the same time without fighting audio feedback (squealing).
It can also be used to reduce environmental noise by canceling it with
digitally generated antinoise.

\section{Audio Processing}

The two principal human senses are vision and hearing. Correspondingly,
much Of DSP is related to image and audio processing. People listen to
both music and speech. DSP has made revolutionary changes in both
these areas.
\subsubsection{Music}
The path leading from the musician's microphone to the audiophile's speaker is
remarkably long. Digital data representation is important to prevent the
degradation commonly associated With analog storage and manipulation. This
is very familiar to anyone Who has compared the musical quality Of cassette
tapes with compact disks. In a typical scenario, a musical piece is recorded in
a sound studio on multiple channels or tracks. In some cases, this even involves
recording individual instruments and singers separately. This is done to give
the sound engineer greater flexibility in creating the final product. The
complex process of combining the individual tracks into a final product is
called mix down. DSP can provide several important functions during mix
down, including: filtering, signal addition and subtraction, signal editing, etc.
One of the most interesting DSP applications in music preparation is
artificial reverberation. If the individual channels are simply added together,
the resulting piece sounds frail and diluted, much as if the musicians were
playing outdoors. This is because listeners are greatly influenced by the echo
or reverberation content Of the music, which is usually minimized in the sound
studio. DSP allows artificial echoes and reverberation to be added during
mix down to simulate various ideal listening environments. Echoes With
delays Of a few hundred milliseconds give the impression Of cathedral like
locations. Adding echoes With delays Of 10-20 milliseconds provide the
perception of more modest size listening rooms,


\subsubsection{Speech generation}
Speech generation and recognition are used to communicate between humans
and machines. Rather than using your hands and eyes, you use your mouth and
ears. This is very convenient when your hands and eyes should be doing
something else, such as: driving a car, performing surgery, or (unfortunately)
firing your weapons at the enemy. Two approaches are used for computer
generated speech: digital recording and vocal tract simulation. In digital
recording, the voice Of a human speaker is digitized and stored, usually in a
compressed form. During playback, the stored data are uncompressed and
converted back into an analog signal. An entire hour Of recorded speech
requires only about three megabytes Of storage, well within the capabilities Of
even small computer systems. This is the most common method Of digital
speech generation used today.
Vocal tract simulators are more complicated, trying to mimic the physical
mechanisms by which humans create speech. The human vocal tract is an
acoustic cavity with resonate frequencies determined by the size and shape of
the chambers. Sound originates in the vocal tract in one of two basic ways,
called voiced and fricative sounds. With voiced sounds, vocal cord vibration
produces near periodic pulses of air into the vocal cavities. In comparison,
fricative sounds originate from the noisy air turbulence at narrow constrictions,
such as the teeth and lips. Vocal tract simulators operate by generating digital
signals that resemble these two types of excitation. The characteristics of the
resonate chamber are simulated by passing the excitation signal through a
digital filter with similar resonances. This approach was used in one of the
very early DSP success stories, the Speak & Spell, a widely sold electronic
learning aid for children.


\subsubsection{Speech recognition}

The automated recognition of human speech is immensely more difficult
than speech generation. Speech recognition is a classic example Of things
that the human brain does well, but digital computers do poorly. Digital
computers can store and recall vast amounts Of data, perform mathematical
calculations at blazing speeds, and do repetitive tasks without becoming
bored or inefficient. Unfortunately, present day computers perform very
poorly when faced With raw sensory data. Teaching a computer to send you
a monthly electric bill is easy. Teaching the same computer to understand
your voice is a major undertaking.
Digital Signal Processing generally approaches the problem of voice
recognition in two steps: feature extraction followed by feature matching.
Each word in the incoming audio signal is isolated and then analyzed to
identify the type of excitation and resonate frequencies. These parameters are
then compared with previous examples of spoken words to identify the closest
match. Often, these systems are limited to only a few hundred words; can
only accept speech With distinct pauses between words; and must be retrained
for each individual speaker. While this is adequate for many commercial applications, these limitations are humbling when compared to the abilities Of
human hearing. There is a great deal of work to be done in this area, with
tremendous financial rewards for those that produce successful commercial
products.

\section{ADC and DAC}

Most of the signals directly encountered in science and engineering are continuous: light intensity
that changes with distance; voltage that varies over time; a chemical reaction rate that depends
on temperature, etc. Analog-to-Digital Conversion (ADC) and Digital-to-Analog Conversion
(DAC) are the processes that allow digital computers to interact with these everyday signals.
Digital information is different from its continuous counterpart in two important respects: it is
sampled, and it is quantized. Both of these restrict how much information a digital signal can
contain. This chapter is about information management: understanding What information you
need to retain, and what information you can afford to lose. In turn, this dictates the selection
of the sampling frequency, number of bits, and type of analog filtering needed for converting
between the analog and digital realms.

\subsection{Quantization}

First, a bit Of trivia. AS you know, it is a digital computer, not a digit
computer. The information processed is called digital data, not digit data.
Why then, is analog-to-digital conversion generally called: dizi.Lize and
diziLization, rather than digitalize and di italization? The answer is nothing
you would expect. When electronics got around to inventing digital techniques,
the preferred names had already been snatched up by the medical community
nearly a century before. Digitalize and digitalization mean to administer the
heart stimulant digitalis.
Figure 3-1 shows the electronic waveforms of a typical analog-to-digital
conversion. Figure (a) is the analog signal to be digitized. As shown by the
labels on the graph, this signal is a voltage that varies over time. To make
the numbers easier, we will assume that the voltage can vary from O to 4.095
volts, corresponding to the digital numbers between O and 4095 that will be
produced by a 12 bit digitizer. Notice that the block diagram is broken into
two sections, the sample-and-hold (S/H), and the analog-to-digital converter
(ADC). As you probably learned in electronics classes, the sample-and-hold
is required to keep the voltage entering the ADC constant while the