\chapter{Introduction to digital signal processingh}

\section{The Roots of DSP}
% \href{https://www.dspguide.com/ch1/1.htm}{test}

Digital Signal Processing is distinguished from other areas in computer science by the unique type of data it uses: signals. In most cases, these signals originate as sensory data from the real world: seismic vibrations, visual images, sound waves, etc. DSP is the mathematics, the algorithms, and the techniques used to manipulate these signals after they have been converted into a digital form. This includes a wide variety of goals, such as: enhancement of visual images, recognition and generation of speech, compression of data for storage and transmission, etc. Suppose we attach an analog-to-digital converter to a computer and use it to acquire a chunk of real world data. DSP answers the question: What next?

The roots of DSP are in the 1960s and 1970s when digital computers first became available. Computers were expensive during this era, and DSP was limited to only a few critical applications. Pioneering efforts were made in four key areas: radar & sonar, where national security was at risk; oil exploration, where large amounts of money could be made; space exploration, where the data are irreplaceable; and medical imaging, where lives could be saved. The personal computer revolution of the 1980s and 1990s caused DSP to exploded with new applications. Rather than being motivated by military and government needs, DSP was suddenly driven by the commercial marketplace. Anyone who thought they could make money in the rapidly expanding field was suddenly a DSP vender. DSP reached the public in such products as: mobile telephones, compact disc players, and electronic voice mail. Figure 1-1 illustrates a few of these varied applications.

This technological revolution occurred from the top-down. In the early 1980s, DSP was taught as a graduate level course in electrical engineering. A decade later, DSP had become a standard part of the undergraduate curriculum. Today, DSP is a basic skill needed by scientists and engineers 


in many fields. As an analogy, DSP can be compared to a previous technological revolution: electronics. While still the realm of electrical engineering, nearly every scientist and engineer has some background in basic circuit design. Without it, they would be lost in the technological world. DSP has the same future.

This recent history is more than a curiosity; it has a tremendous impact on your ability to learn and use DSP. Suppose you encounter a DSP problem, and turn to textbooks or other publications to find a solution. What you will typically find is page after page of equations, obscure mathematical symbols, and unfamiliar terminology. It's a nightmare! Much of the DSP literature is baffling even to those experienced in the field. It's not that there is anything wrong with this material, it is just intended for a very specialized audience. State-of-the-art researchers need this kind of detailed mathematics to understand the theoretical implications of the work.

A basic premise of this book is that most practical DSP techniques can be learned and used without the traditional barriers of detailed mathematics and theory. The Scientist and Engineers Guide to Digital Signal Processing is written for those who want to use DSP as a tool, not a new career.

The remainder of this chapter illustrates areas where DSP has produced revolutionary changes. As you go through each application, notice that DSP is very interdisciplinary, relying on the technical work in many adjacent fields. As Fig. 1-2 suggests, the borders between DSP and other technical disciplines are not sharp and well defined, but rather fuzzy and overlapping. If you want to specialize in DSP, these are the allied areas you will also need to study. 

\sections{Basic Concept}

Digital Signal Processing is one of the most powerful technologies that will shape science and engineering in the twenty-first century. Revolutionary changes have already been made in a broad range of fields: communications, medical imaging, radar & sonar, high fidelity music reproduction, and oil prospecting, to name just a few. Each of these areas has developed a deep DSP technology, with its own algorithms, mathematics, and specialized techniques. This combination of breath and depth makes it impossible for any one individual to master all of the DSP technology that has been developed. DSP education involves two tasks: learning general concepts that apply to the field as a whole, and learning specialized techniques for your particular area of interest. This chapter starts our journey into the world of Digital Signal Processing by describing the dramatic effect that DSP has made in several diverse fields. The revolution has begun.


\subsubsection{Signal Frequency (Spectrum) Analysis}

\subsection{Overview of Typical Digital Signal Processing in Real-World Applications}

\subsubsection{Digital Crossover Audio System}

\subsubsection{Interference Cancellation in Electrocardiography}

\subsubsection{Speech Coding and Compression}

\subsubsection{Vibration Signature Analysis for Defected Gear Teeth}

\subsubsection{Digital Image Enhancement}

\subsubsection{Artificial Intelligence Examples}

\subsection{Digital Signal Processing Applications}

\subsection{Summary}
